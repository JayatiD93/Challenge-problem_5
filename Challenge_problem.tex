\documentclass[journal,12pt,twocolumn]{IEEEtran}
%
\usepackage{setspace}
\usepackage{gensymb}
%\doublespacing
\singlespacing

%\usepackage{graphicx}
%\usepackage{amssymb}
%\usepackage{relsize}
\usepackage[cmex10]{amsmath}
%\usepackage{amsthm}
%\interdisplaylinepenalty=2500
%\savesymbol{iint}
%\usepackage{txfonts}
%\restoresymbol{TXF}{iint}
%\usepackage{wasysym}
\usepackage{amsthm}
%\usepackage{iithtlc}
\usepackage{mathrsfs}
\usepackage{txfonts}
\usepackage{stfloats}
\usepackage{bm}
\usepackage{cite}
\usepackage{cases}
\usepackage{subfig}
%\usepackage{xtab}
\usepackage{longtable}
\usepackage{multirow}
%\usepackage{algorithm}
%\usepackage{algpseudocode}
\usepackage{enumitem}
\usepackage{mathtools}
\usepackage{steinmetz}
\usepackage{tikz}
\usepackage{circuitikz}
\usepackage{verbatim}
\usepackage{tfrupee}
\usepackage[breaklinks=true]{hyperref}
%\usepackage{stmaryrd}
\usepackage{tkz-euclide} % loads  TikZ and tkz-base
%\usetkzobj{all}
\usetikzlibrary{calc,math}
\usepackage{listings}
    \usepackage{color}                                            %%
    \usepackage{array}                                            %%
    \usepackage{longtable}                                        %%
    \usepackage{calc}                                             %%
    \usepackage{multirow}                                         %%
    \usepackage{hhline}                                           %%
    \usepackage{ifthen}                                           %%
  %optionally (for landscape tables embedded in another document): %%
    \usepackage{lscape}     
\usepackage{multicol}
\usepackage{chngcntr}
%\usepackage{enumerate}

%\usepackage{wasysym}
%\newcounter{MYtempeqncnt}
\DeclareMathOperator*{\Res}{Res}
%\renewcommand{\baselinestretch}{2}
\renewcommand\thesection{\arabic{section}}
\renewcommand\thesubsection{\thesection.\arabic{subsection}}
\renewcommand\thesubsubsection{\thesubsection.\arabic{subsubsection}}

\renewcommand\thesectiondis{\arabic{section}}
\renewcommand\thesubsectiondis{\thesectiondis.\arabic{subsection}}
\renewcommand\thesubsubsectiondis{\thesubsectiondis.\arabic{subsubsection}}

% correct bad hyphenation here
\hyphenation{op-tical net-works semi-conduc-tor}
\def\inputGnumericTable{}                                 %%

\lstset{
%language=C,
frame=single, 
breaklines=true,
columns=fullflexible
}
%\lstset{
%language=tex,
%frame=single, 
%breaklines=true
%}

\begin{document}
%


\newtheorem{theorem}{Theorem}[section]
\newtheorem{problem}{Problem}
\newtheorem{proposition}{Proposition}[section]
\newtheorem{lemma}{Lemma}[section]
\newtheorem{corollary}[theorem]{Corollary}
\newtheorem{example}{Example}[section]
\newtheorem{definition}[problem]{Definition}
%\newtheorem{thm}{Theorem}[section] 
%\newtheorem{defn}[thm]{Definition}
%\newtheorem{algorithm}{Algorithm}[section]
%\newtheorem{cor}{Corollary}
\newcommand{\BEQA}{\begin{eqnarray}}
\newcommand{\EEQA}{\end{eqnarray}}
\newcommand{\define}{\stackrel{\triangle}{=}}

\bibliographystyle{IEEEtran}
%\bibliographystyle{ieeetr}


\providecommand{\mbf}{\mathbf}
\providecommand{\pr}[1]{\ensuremath{\Pr\left(#1\right)}}
\providecommand{\qfunc}[1]{\ensuremath{Q\left(#1\right)}}
\providecommand{\sbrak}[1]{\ensuremath{{}\left[#1\right]}}
\providecommand{\lsbrak}[1]{\ensuremath{{}\left[#1\right.}}
\providecommand{\rsbrak}[1]{\ensuremath{{}\left.#1\right]}}
\providecommand{\brak}[1]{\ensuremath{\left(#1\right)}}
\providecommand{\lbrak}[1]{\ensuremath{\left(#1\right.}}
\providecommand{\rbrak}[1]{\ensuremath{\left.#1\right)}}
\providecommand{\cbrak}[1]{\ensuremath{\left\{#1\right\}}}
\providecommand{\lcbrak}[1]{\ensuremath{\left\{#1\right.}}
\providecommand{\rcbrak}[1]{\ensuremath{\left.#1\right\}}}
\theoremstyle{remark}
\newtheorem{rem}{Remark}
\newcommand{\sgn}{\mathop{\mathrm{sgn}}}
\providecommand{\abs}[1]{\left\vert#1\right\vert}
\providecommand{\res}[1]{\Res\displaylimits_{#1}} 
\providecommand{\norm}[1]{\left\lVert#1\right\rVert}
%\providecommand{\norm}[1]{\lVert#1\rVert}
\providecommand{\mtx}[1]{\mathbf{#1}}
\providecommand{\mean}[1]{E\left[ #1 \right]}
\providecommand{\fourier}{\overset{\mathcal{F}}{ \rightleftharpoons}}
%\providecommand{\hilbert}{\overset{\mathcal{H}}{ \rightleftharpoons}}
\providecommand{\system}{\overset{\mathcal{H}}{ \longleftrightarrow}}
	%\newcommand{\solution}[2]{\textbf{Solution:}{#1}}
\newcommand{\solution}{\noindent \textbf{Solution: }}
\newcommand{\cosec}{\,\text{cosec}\,}
\providecommand{\dec}[2]{\ensuremath{\overset{#1}{\underset{#2}{\gtrless}}}}
\newcommand{\myvec}[1]{\ensuremath{\begin{pmatrix}#1\end{pmatrix}}}
\newcommand{\mydet}[1]{\ensuremath{\begin{vmatrix}#1\end{vmatrix}}}
%\numberwithin{equation}{section}
\numberwithin{equation}{subsection}
%\numberwithin{problem}{section}
%\numberwithin{definition}{section}
\makeatletter
\@addtoreset{figure}{problem}
\makeatother

\let\StandardTheFigure\thefigure
\let\vec\mathbf
%\renewcommand{\thefigure}{\theproblem.\arabic{figure}}
\renewcommand{\thefigure}{\theproblem}
%\setlist[enumerate,1]{before=\renewcommand\theequation{\theenumi.\arabic{equation}}
%\counterwithin{equation}{enumi}


%\renewcommand{\theequation}{\arabic{subsection}.\arabic{equation}}

\def\putbox#1#2#3{\makebox[0in][l]{\makebox[#1][l]{}\raisebox{\baselineskip}[0in][0in]{\raisebox{#2}[0in][0in]{#3}}}}
     \def\rightbox#1{\makebox[0in][r]{#1}}
     \def\centbox#1{\makebox[0in]{#1}}
     \def\topbox#1{\raisebox{-\baselineskip}[0in][0in]{#1}}
     \def\midbox#1{\raisebox{-0.5\baselineskip}[0in][0in]{#1}}

\vspace{3cm}


\title{Challenge Problem 5}
\author{Jayati Dutta}





% make the title area
\maketitle

\newpage

%\tableofcontents

\bigskip

\renewcommand{\thefigure}{\theenumi}
\renewcommand{\thetable}{\theenumi}
%\renewcommand{\theequation}{\theenumi}


%\begin{abstract}
%This is a simple document explaining how to determine the QR decomposition of a 2x2 matrix.
%\end{abstract}

%Download all python codes 
%
%\begin{lstlisting}
%svn co https://github.com/JayatiD93/trunk/My_solution_design/codes
%\end{lstlisting}

%Download all and latex-tikz codes from 
%%
%\begin{lstlisting}
%svn co https://github.com/gadepall/school/trunk/ncert/geometry/figs
%\end{lstlisting}
%%


\section{Problem}
Challenging problem: Intersection of transforms, matrices, signal processing, probability: 
 
\begin{lstlisting}
https://github.com/abhishekt711/EE5609/blob/master/Assignment20/Assignment_20.pdf
\end{lstlisting}

This problem provides an opportunity to link all the above. What is the link?

\section{Explanation}

The transition probability of Markov chain is expressed as:\\
$P_{ij}$ = P[next state $S_j$ at time t=1$|$current state $S_i$ at t=0]\\
and the Transition Probability Matrix (TPM) is given as:
\begin{align}
P = \myvec{P_{11} & P_{12} & P_{13}\\P_{21} & P_{22} & P_{23}\\P_{31} & P_{32} & P_{33}}
\end{align}
In this problem, $P_{11}=P_{22}=P_{33}=0 \implies$ Any state cannot retain in its own state after 1 step execution, that is , for n=1. These $P_{11},P_{22},P_{33}$ are theretaintion probabilities.


The summamtion of each row of this TPM, $P$ is 1 which is similar to a Binary Symmetric Channel Matrix which is very important concept in Information Theory.

Now, while considering 2nd step execution, the TPM for n=2 is:
\begin{align}
P = \myvec{\frac{1}{2} & \frac{1}{4} & \frac{1}{4}\\\frac{1}{4} & \frac{1}{2} & \frac{1}{4}\\\frac{1}{4} & \frac{1}{4} & \frac{1}{2}}
\end{align}
From here we get that after 2nd execution, the retaintion probability is $\frac{1}{2}$ for all the 3 states, that is,\\
$P_{11}^{(2)}$ = P[next state $S_1$ at time t=2$|$current state $S_1$ at t=0] = $\frac{1}{2}$.\\

The transition probabilities from any state to other states are equal, that is, $P_{21}=P_{23}$ ,$P_{31}=P_{32}$ and $P_{12}=P_{13}$.

The transition probability from one state to another state of a current state is the retaintion probability of the next state, that is, $P_{21}^{(n)}=P_{23}^{(n)}=P_{22}^{(n+1)}$, for all the 3 states.

In each period , that is, for n=0,1,2,..., the retaintion probabilities($P_{22}, P_{22}^{(2)}, P_{22}^{(3)},....$) for all 3 states are same, that is, $P_{11}^{(2)}=P_{22}^{(2)}=P_{33}^{(2)}$.

$\vec{\pi}$ is the stationary distribution and $\vec{\pi}=\myvec{\pi_1\\\pi_2\\\pi_3}$ such that $\vec{\pi P}=\vec{\pi} \implies$ The markov chain is in steady state with period 1 and $\vec{\pi}$ is the state probabilities vector which is fixed for the entire Markov chain with the values $\vec{\pi}=\myvec{\frac{1}{3}\\\frac{1}{3}\\\frac{1}{3}}$. It means the Markov chain is in steady state conditionand all the 3 states have equal share in the process.

If we consider $n=n_1$th execution, then the output of the state 1 can be defined as:
\begin{align}
Y_1[n_1]= \frac{1}{2}Y_2[n_1-1]+ \frac{1}{2}Y_3[n_1-1]
\end{align}
Where $Y_1, Y_2, Y_3$ denote the output of state 1, state 2, state 3. Similarly,
\begin{align}
Y_2[n_1]= \frac{1}{2}Y_1[n_1-1]+ \frac{1}{2}Y_3[n_1-1]\\
Y_3[n_1]= \frac{1}{2}Y_1[n_1-1]+ \frac{1}{2}Y_2[n_1-1]
\end{align}
It works as a Finite state machine. Markov chain can be represented by Finite State Machine(FSM) as the states of time t+1 depends on the states of time t. But in the Markov chain the transition of the states are probabilistic while in FSM it is deterministic.

%\renewcommand{\theequation}{\theenumi}
%\begin{enumerate}[label=\thesection.\arabic*.,ref=\thesection.\theenumi]
%\numberwithin{equation}{enumi}
%\item Verification of the above problem using python code.\\
%\solution The  following Python code generates Fig. \ref{fig:parabola}
%\begin{lstlisting}
%codes/check_parab.py
%\end{lstlisting}
%
%So the solution is matching with the given plot, hence it is verified.
%%
%\end{enumerate}

\end{document}



